\documentclass[landscape,a0paper,fontscale=0.45,margin=1cm]{baposter/baposter}
%\documentclass[landscape,paperwidth=1682mm,paperheight=1189mm,fontscale=0.4,margin=3cm]{baposter}

\usepackage{calc}
\usepackage{graphicx}
\usepackage{amsmath}
\usepackage{amssymb}
\usepackage{relsize}
\usepackage{multirow}
\usepackage{rotating}
\usepackage{bm}
\usepackage{enumitem}
\usepackage{booktabs}
\usepackage{relsize}		% For \smaller
\usepackage{url}			% For \url

\usepackage{graphicx}
\usepackage{multicol}
\usepackage{subcaption}
\captionsetup{compatibility=false}

\usepackage{nameref}
%\usepackage{times}
%\usepackage{helvet}
%\usepackage{bookman}
\usepackage{palatino}
%% Pablo's personal packages
%% AMS unofficial LateX proceeding template:
%%    http://www.cimms.ou.edu/~lakshman/ametsoc/
%%\usepackage[hypertex]{hyperref}
%% end personal packages

%\newcommand{\captionfont}{\footnotesize}

\graphicspath{{images/}{./images}}
\usetikzlibrary{calc}


\newcommand{\Matrix}[1]{\begin{bmatrix} #1 \end{bmatrix}}
\newcommand{\Vector}[1]{\begin{pmatrix} #1 \end{pmatrix}}

\newcommand*{\norm}[1]{\mathopen\| #1 \mathclose\|}% use instead of $\|x\|$
\newcommand*{\abs}[1]{\mathopen| #1 \mathclose|}% use instead of $\|x\|$
\newcommand*{\normLR}[1]{\left\| #1 \right\|}% use instead of $\|x\|$

\newcommand*{\SET}[1]  {\ensuremath{\mathcal{#1}}}
\newcommand*{\FUN}[1]  {\ensuremath{\mathcal{#1}}}
\newcommand*{\MAT}[1]  {\ensuremath{\boldsymbol{#1}}}
\newcommand*{\VEC}[1]  {\ensuremath{\boldsymbol{#1}}}
\newcommand*{\CONST}[1]{\ensuremath{\mathit{#1}}}

\DeclareMathOperator*{\argmax}{arg\,max}
\DeclareMathOperator*{\diag}{diag}
\DeclareMathOperator*{\argmin}{arg\,min}
\DeclareMathOperator*{\vectorize}{vec}
\DeclareMathOperator*{\reshape}{reshape}

%\font\dsfnt=dsrom12

\newcommand{\SNN}{\ensuremath{\mathbb N}}
\newcommand{\SRR}{\ensuremath{\mathbb R}}
\newcommand{\SZZ}{\ensuremath{\mathbb Z}}
%-----------------------------------------------------------------------------
% Matrices of the shape model
\renewcommand{\a}{\VEC\alpha}
\renewcommand{\v}{\VEC v}
\renewcommand{\l}{\VEC l}
\newcommand*{\m}{\VEC{\mu}}
\newcommand*{\M}{\MAT{M}}
\renewcommand*{\P}{\MAT{\Pi}}

%\newcommand{\J}{\SET J}
\newcommand{\J}{\SET{P}}
\newcommand{\Active}{\mathcal{A}}
\newcommand{\Selection}{\mathbf{S}}
\newcommand{\AllSelections}{\mathfrak{S}}
\newcommand{\Params}{\VEC\Theta}

%%%%%%%%%%%%%%%%%%%%%%%%%%%%%%%%%%%%%%%%%%%%%%%%%%%%%%%%%%%%%%%%%%%%%%%%%%%%%%%%
%%%% Some math symbols used in the text
%%%%%%%%%%%%%%%%%%%%%%%%%%%%%%%%%%%%%%%%%%%%%%%%%%%%%%%%%%%%%%%%%%%%%%%%%%%%%%%%

%%%%%%%%%%%%%%%%%%%%%%%%%%%%%%%%%%%%%%%%%%%%%%%%%%%%%%%%%%%%%%%%%%%%%%%%%%%%%%%%
% Multicol Settings
%%%%%%%%%%%%%%%%%%%%%%%%%%%%%%%%%%%%%%%%%%%%%%%%%%%%%%%%%%%%%%%%%%%%%%%%%%%%%%%%
\setlength{\columnsep}{1.5em}
\setlength{\columnseprule}{0mm}

%%%%%%%%%%%%%%%%%%%%%%%%%%%%%%%%%%%%%%%%%%%%%%%%%%%%%%%%%%%%%%%%%%%%%%%%%%%%%%%%
% Save space in lists. Use this after the opening of the list
%%%%%%%%%%%%%%%%%%%%%%%%%%%%%%%%%%%%%%%%%%%%%%%%%%%%%%%%%%%%%%%%%%%%%%%%%%%%%%%%
\newcommand{\compresslist}{%
\setlength{\itemsep}{1pt}%
\setlength{\parskip}{0pt}%
\setlength{\parsep}{0pt}%
}

%%%%%%%%%%%%%%%%%%%%%%%%%%%%%%%%%%%%%%%%%%%%%%%%%%%%%%%%%%%%%%%%%%%%%%%%%%%%%%
%%% Begin of Document
%%%%%%%%%%%%%%%%%%%%%%%%%%%%%%%%%%%%%%%%%%%%%%%%%%%%%%%%%%%%%%%%%%%%%%%%%%%%%%

\begin{document}

%%%%%%%%%%%%%%%%%%%%%%%%%%%%%%%%%%%%%%%%%%%%%%%%%%%%%%%%%%%%%%%%%%%%%%%%%%%%%%
%%% Here starts the poster
%%%---------------------------------------------------------------------------
%%% Format it to your taste with the options
%%%%%%%%%%%%%%%%%%%%%%%%%%%%%%%%%%%%%%%%%%%%%%%%%%%%%%%%%%%%%%%%%%%%%%%%%%%%%%
% Define some colors

\definecolor{silver}{cmyk}{0,0,0,0.3}
\definecolor{yellow}{cmyk}{0,0,0.9,0.0}
\definecolor{reddishyellow}{cmyk}{0,0.22,1.0,0.0}
\definecolor{black}{cmyk}{0,0,0.0,1.0}
\definecolor{white}{rgb}{1,1,1}
\definecolor{red}{rgb}{.9,0,0}
\definecolor{green}{rgb}{0,.3,0}
\definecolor{blue}{rgb}{.21,.24,.36}

\definecolor{darkYellow}{cmyk}{0,0,1.0,0.5}
\definecolor{darkSilver}{cmyk}{0,0,0,0.1}

\definecolor{middlegray}{rgb}{.4,.4,.4}
\definecolor{lightgray}{rgb}{.7,.7,.7}

\definecolor{lightgreen}{rgb}{.7,7,.35}
\definecolor{lightlila}{rgb}{.6,.6,.8}
\definecolor{lightorange}{rgb}{.88,.74,.15}
\definecolor{lighterorange}{rgb}{.98,.84,.45}
\definecolor{middleblue}{rgb}{.447,.537,.9513}
\definecolor{lightblue}{rgb}{.447,.537,.6313}
\definecolor{lighterblue}{rgb}{.568,.639,.709}
\definecolor{lighteryellow}{cmyk}{0,0,0.1,0.0}
\definecolor{lightestblue}{rgb}{.6,.7,.8}

%%% Setting Background Image %%%%%%%%%%%%%%%%%%%%%%%%%%%%%%%%%%%%%%%%%%%%%%%%%%
\background{
	\begin{tikzpicture}[remember picture,overlay]%
	\draw (current page.south west)+(-1em,-1em) node[anchor=south west]
	{\includegraphics[width=1.01\linewidth,height=0.23\textheight]{baposter/uni_leipzig-bg_quer.png}};
	\end{tikzpicture}
}


%\hyphenation{resolution occlusions}
%%
\begin{poster}%
  % Poster Options
  {
  % Show grid to help with alignment
  grid=false, %true,
  % Number of columns
  columns=4,
  % Column spacing
  colspacing=1.0em,
  % Color style
  bgColorOne= red, %white, %lightgreen, %silver,
  bgColorTwo= white, %white, %middlegray, %lightestblue, %white,
  borderColor=gray, %reddishyellow, %lightorange,
  headerColorOne=darkSilver, %lighterblue, %lightblue,   %lightblue,
  headerColorTwo=lightgray, 
  headerFontColor=black, %lightorange, %black,
  boxColorOne=darkSilver, %darkSilver, %darkblue, %darkYellow,
  boxColorTwo=red, %white,
  % Format of textbox
  textborder=faded,
  % Format of text header
  eyecatcher=true,
  headerborder=none, %open, %none,
  headerheight=0.19\textheight,
  %textfont=\sc, %An example of changing the text font
  headershape=rounded, %smallrounded,
  headershade=shadeLR,
  headerfont=\LARGE\bf,  %%\Large\bf\textsc, %Sans Serif
  textfont={\color{black}\setlength{\parindent}{1.5em}},
  boxshade=none, %shadeTB, %shadeLR,
  background=user, %shadeTB,
  linewidth=2.5pt
  }
  % Eye Catcher
  {
      \includegraphics[height=15.0em]{baposter/uni_leipzig-logo.png}\\
      %\vspace{+3em}
      %\includegraphics[height=6.0em]{logo_small-ac3.png}
      %\hspace{3em}
      %\includegraphics[height=3.0em]{PosterNumber.png}
  }
  % Title
  {
  	Cloud Macro- and Microphysical Properties as Coupled to Sea Ice Leads During\\the MOSAiC Expedition
  }
  % Authors
  {\vspace{+1em} \color{black}Pablo Saavedra~Garfias$^{1,*}$, Heike Kalesse-Los$^{1}$, Luisa von~Albedyll$^{2}$, Hannes Griesche$^{3}$, Gunnar Spreen$^{4}$\\
    $^1$University of Leipzig, Institute for Meteorology, Faculty of Physics and Geosciences\\
    %$^2$University of Cologne, Institute of Geophysics and Meteorology\\
    $^2$Alfred Wegener Institute, Helmholtz Centre for Polar and Marine Research (AWI)\\
    $^3$Leipzig Institute for Tropospheric Research (TROPOS)\\
    $^4$University of Bremen, Institute of Environmental Physics\\
	$^*$Contact: \url{pablo.saavedra@uni-leipzig.de}
    %% {\color{blue} \url{www2.meteo.uni-bonn.de/admirari}}
   }
  % University logo
  {% The makebox allows the title to flow into the logo, this is a hack
   % because of the L shaped logo.
    \begin{tabular}{r}
   	\includegraphics[height=7.1em]{agu_logo.png}\\
   	\vspace{+10em}\\
    \includegraphics[height=5.1em]{logo_small-ac3.png}
    \end{tabular}   
  }

%%%%%%%%%%%%%%%%%%%%%%%%%%%%%%%%%%%%%%%%%%%%%%%%%%%%%%%%%%%%%%%%%%%%%%%%%%%%%%
%%% Now define the boxes that make up the poster
%%%---------------------------------------------------------------------------
%%% Each box has a name and can be placed absolutely or relatively.
%%% The only inconvenience is that you can only specify a relative position 
%%% towards an already declared box. So if you have a box attached to the 
%%% bottom, one to the top and a third one which should be in between, you 
%%% have to specify the top and bottom boxes before you specify the middle 
%%% box.
%%%%%%%%%%%%%%%%%%%%%%%%%%%%%%%%%%%%%%%%%%%%%%%%%%%%%%%%%%%%%%%%%%%%%%%%%%%%%%
    %
    % A coloured circle useful as a bullet with an adjustably strong filling
    \newcommand{\colouredcircle}{%
      \vspace{1em}\tikz{\useasboundingbox (-0.2em,-0.32em) rectangle(0.2em,0.32em); \draw[draw=black,fill=lightblue,line width=0.03em] (0,0) circle(0.28em);}\hspace{0.8em}}

% PSG: my ancillary commands:
\newcommand{\polarstern}{RV~\emph{Polarstern}\,}
\newcommand{\mosaic}{\emph{MOSAiC}\,}
\newcommand{\refp}[1]{[\ref{#1}]}
% ----/

%%%%%%%%%%%%%%%%%%%%%%%%%%%%%%%%%%%%%%%%%%%%%%%%%%%%%%%%%%%%%%%%%%%%%%%%%%%%%%
  \headerbox{1.- Research Objectives}{name=objective,column=0,span=1,row=0}{
%%%%%%%%%%%%%%%%%%%%%%%%%%%%%%%%%%%%%%%%%%%%%%%%%%%%%%%%%%%%%%%%%%%%%%%%%%%%%%
The study focuses on the observation of Arctic mixes-phase clouds and sea ice leads to address the following research questions:

\begin{tabular*}{0.99\textwidth}[h!]{lr}
	\begin{minipage}{0.68\textwidth}
		\vspace{-4em}
		\begin{itemize}
			\item Are cloud properties influenced by the presence of sea ice leads?
			\item Does coupling/decoupling of clouds to moisture-layers impact the cloud's properties?
		\end{itemize}
		
	{\small We focus is wintertime/early spring legs 1 to 3 of the \mosaic expedition~\refp{bib:Shupe2022}. Instrumentation and data set are provided by the Atmospheric Radiation Measurement’s (ARM) Mobile Facility 1 (AMF-1) and by the OCEANET-Atmosphere container from TROPOS.}
	\end{minipage}
	&
	\includegraphics[scale=0.4]{arctic_Pstern_drift20192020.png}	
\end{tabular*}
\\
%The case study of 18 Nov 2019 is used to highlight the research methods.
  }
%%%%%%%%%%%%%%%%%%%%%%%%%%%%%%%%%%%%%%%%%%%%%%%%%%%%%%%%%%%%%%%%%%%%%%%%%%%%%%
\headerbox{2.- Coupling of Sea Ice and Clouds}{name=methods,column=0,below=objective}{
%%%%%%%%%%%%%%%%%%%%%%%%%%%%%%%%%%%%%%%%%%%%%%%%%%%%%%%%%%%%%%%%%%%%%%%%%%%%%%
%\hspace{-2em}
\begin{minipage}{0.98\textwidth}
	\centering
	\includegraphics[width=.9\linewidth]{new_drawing_lead_cloud.png}
	\captionof{figure}{\smaller Sea ice interaction with observed clouds. Adapted from \refp{bib:jannos2022}}
	\label{fig:leadcloud}
\end{minipage}\\

Daily sea ice lead fraction (LF) is obtained based on the divergence calculations from consecutive Sentinel-1 SAR scenes~\refp{bib:vonAlbedyll2021}. Sea ice concentration (SIC) is provided by the University of Bremen~\refp{bib:Ludwig2021}. Fig.~\ref{fig:lf_sic} summarizes the LF and SIC during \mosaic wintertime.
\begin{minipage}{0.99\textwidth}
	\centering
		\includegraphics[width=.5\linewidth]{bias_sic_timeseries.png}\hspace{1cm}
		\includegraphics[width=.3\linewidth]{mosaic0152_20191115T03271573_20191118T05301574_LF.png}
	\captionsetup{width=0.8\linewidth}
	\captionof{figure}{\small Left: LF and SIC, vertical dashed-grey lines mark the Sentinel-1 data gap. Right: case study 18 Nov 2019.}
	\label{fig:lf_sic}
\end{minipage}

We relate sea ice lead fraction to cloud observations above \polarstern following:

\colouredcircle LF products is analyzed within 50~km around the \polarstern (red star in Fig.~\ref{fig:lf_sic}, right) with updated coordinates every minute.\\

\colouredcircle Sea ice - atmosphere coupling conceptual model\\
Vertical gradient of water vapour transport ($\nabla WVT$) is calculated from specific humidity $q_v~\mathrm{[g~g^{-1}]}$ and horizontal wind $\vec{v}_w~\mathrm{[m~s^{-1}]}$ from radiosonde profiles, following
\begin{equation}
	\nabla \mathrm{WVT} = -\frac{10^2}{g}~|q_v\cdot \vec{v}_w|~\frac{dP}{dz}
\end{equation}
The direction of maximum transport (see grey lines in Fig.~\ref{fig:lf_sic}) is used to relate LF with zenith observations at \polarstern.

\colouredcircle Planetary boundary layer height (PBLH)\\
Estimated via the bulk Richardson number \ref{eq:Rib}, PBLH is used as top layer below which the maximum $\nabla WVT$ is localized:
\begin{equation}
	Ri_b(z)\,=\,\frac{g}{\theta_v}\frac{\Delta \theta_v \,\Delta z}{\left( \Delta u \right)^2\,+\,\left(\Delta v \right)^2}
	\label{eq:Rib}
\end{equation}
%with $g$ is the constant of gravity, $\theta_v$ is the virtual potential temperature profile in K, $\Delta \theta_v\,=\,\theta_v-\theta_v(z_0)$, $\Delta u\,=\,u-u_0$, and $\Delta v\,=\,v-v_0$, the horizontal wind components in $\rm m~s^{-1}$. The $\Delta z\,=\,z - z_0$ with $z$ the altitude of the atmosphere layers in $\rm m$ and the subscript $0$ indicating the surface reference.

%\colouredcircle Cloud coupling classification: criteria based on the virtual potential temperature $\theta_v$ and location of maximum $\nabla~WVT$ below PBLH. The $\theta_v$ is analyzed to classify cases where the WVT is coupled or decoupled to the cloud.
%\begin{minipage}{0.98\linewidth}
%	\centering
%		\includegraphics[width=0.32\linewidth]{high_coupling_profile_1243UTC.png}
%		\includegraphics[width=0.32\linewidth]{low_coupling_profile_1417UTC.png}
%		\includegraphics[width=0.32\linewidth]{decoupling_profile_1854UTC.png}
%	\captionof{figure}{\smaller Examples of cloud coupling (left \& middle) and decoupling (right).}
%	\label{fig:cop-dec}
%\end{minipage}

} % end of box

%%%%%%%%%%%%%%%%%%%%%%%%%%%%%%%%%%%%%%%%%%%%%%%%%%%%%%%%%%%%%%%%%%%%%%%%%%%%%%
  \headerbox{3.- Cloud-sea ice coupled case study 18th Nov 2019}{name=results,column=1,row=0,span=2}{
%%%%%%%%%%%%%%%%%%%%%%%%%%%%%%%%%%%%%%%%%%%%%%%%%%%%%%%%%%%%%%%%%%%%%%%%%%%%%%
{\Large Cloudnet target classification is used to determine cloud macro- and microphysical properties. Radiosondes are used to obtain information on the thermodynamic states of the atmosphere, e.g. $\theta_v$, $\nabla WVT$, wind vectors, and $Ri_b$.}\\
\vspace{+2.5em}
\begin{tabular}{ccc} %{@{}lcr@{}}
	\multicolumn{3}{l}{\vspace{+1em}\colouredcircle {\large Synergy of the ship-based zenith observations are needed to apply the Cloudnet classification algorithm.\hfill}} \\
	%\textbf{\Large COUPLED}	& \textbf{\Large DECOUPLED} \\
	\begin{minipage}{0.3\linewidth}
		\begin{center}
			\includegraphics[width=.95\linewidth]{cloudnet_measurements_20191118.png}\\
			\captionsetup{width=0.8\linewidth}
			\captionof{figure}{From top to bottom: ARM KAZR cloud radar reflectivity, ARM ceilometer backscattering coefficient, liquid water path from HATPRO microwave radiometer~\refp{bib:Ebell2022}.}
			\label{fig:measurements}
		\end{center}
	\end{minipage}
	&
	\begin{minipage}{0.33\linewidth}	
		\begin{center}             
			\includegraphics[width=.87\linewidth]{cloudnet_classific_20191118.png}\\
			\includegraphics[width=.91\linewidth]{IWP_LWP_timeseries_20191118}
			\captionsetup{width=0.9\linewidth}
			\captionof{figure}{Top: Cloudnet classification from the measurements in Fig.~\ref{fig:measurements}. Bottom: LWP and IWP for the lowest layer detected. Note that only of mixed-phase clouds are considered.}
			\label{fig:classification}
		\end{center}
	\end{minipage}
	&
	\begin{minipage}{0.33\linewidth}
		\begin{center}
			\includegraphics[width=.77\linewidth]{bulk_richardson_number_20191118.png}\\
			\includegraphics[width=.7\linewidth]{PBLH_zoom_cloudnet_classific_20191118}
			\captionsetup{width=0.75\linewidth}
			\captionof{figure}{Top: $Ri_b$ for lowest 1.5~km, PBLH \@ critical $Ri_b$=1. RS denotes times of radiosonde launches. Bottom: Close-up of Fig.~\ref{fig:classification} with PBLH (dashed-light-green), max~$\nabla WVT$ (green), and cloud bottom and top heights (black lines), and cloud base by the ceilometer (dotted-grey). Coupled status is shown along the x-axis.}
			\label{fig:closeup}
		\end{center}
	\end{minipage}	
\end{tabular}\\
	%\multicolumn{2}{l}{\vspace{+5em}
%	\colouredcircle {\larger Cloud base and -top height (CBH and CTH, respectively) are determined only for mixed-phase clouds for up to three cloud layers from the Cloudnet target classification class (Fig.~\ref{fig:classification}, top)}.\\ % To estimate CBH the Cloudnet pixels with liquid droplets or super-cooled liquid (SCL) are considered, for CTH the ice pixels are used in additon (Fig.~\ref{fig:closeup}).\\
The wind direction at max~$\nabla WVT$ provides the relevant information to link sea ice LF to the cloud observation above \polarstern. LF is considered from a region determined by the wind direction with center at \polarstern to 50~km radius (grey lines in Fig.~\ref{fig:lf_sic}, right).\\

\begin{tabular}{cc}
	\begin{minipage}{0.32\linewidth}
		
		\begin{center}
			\includegraphics[width=.92\linewidth]{leadfraction_sic_wvtdir_18112019}	
			\captionof{figure}{LF extracted from Fig.~\ref{fig:lf_sic} (right) based on 1-minute wind direction at the max $\nabla WVT$. For reference the wind vectors at max $\nabla WVT$ (top panel) and SIC for the same region is also shown in light-blue (right y-axis).}
			\label{fig:wdirwvt}
		\end{center}
		
	\end{minipage}
	&
	\begin{minipage}{0.64\linewidth}
		\centering
		From Fig.~\ref{fig:wdirwvt} the 1-minute LF statistics can be related to the corresponding micro- and macrophysical properties of clouds derived from Cloudnet. In order to reduce variability the following results are averaged in 15~minutes intervals i.e. every point represents $\approx$ 15 observations and bars are their variance.\\
		[a]{\includegraphics[width=.28\linewidth]{LWP_LF_20191118}}\hspace{1cm}
		[b]{\includegraphics[width=.28\linewidth]{IWP_LF_20191118}}
		[c]{\includegraphics[width=.28\linewidth]{LPRcloud_LF_20191118.png}}
		\captionsetup{width=0.8\linewidth}
		\captionof{figure}{[a]~mean single cloud layer LWP vs. LF (black-line in Fig.~\ref{fig:wdirwvt}) with colour-coded cloud top temperature. [b]~Same but for IWP of same cloud layer. [c]~$\Gamma_{\rm cloud}$ as defined in Eq.~\ref{eq:lapserate} vs. LF with colour-coded cloud thickness.}
		\label{fig:scatterWPLF}
	\end{minipage}\\
	&
	\begin{minipage}{0.31\linewidth}
		\begin{equation}
			\Gamma_{\rm cloud}=\frac{\Delta T}{\Delta H}\,=\,\frac{T_{top}-T_{base}}{CTH-CBH}
			\label{eq:lapserate}
		\end{equation}
	\end{minipage}
	\\
	\multicolumn{2}{l}{
		\vspace{+15em}
	\begin{minipage}{0.5\linewidth}
	Fig.~\ref{fig:scatterWPLF} [c] shows the gradient of cloud temperature defined as Eq.~\ref{eq:lapserate}. The most negative $\Gamma_{\rm cloud}$ are close to a moist adiabatic lapse-rate. Positive values indicate a temperature inversion at cloud top. %, however this is not consistent neither with increasing LF nor cloud thickness since for coupled cases there are positive and negative $\Gamma_{\rm cloud}$ which is an indicator of other factors playing a role.
	%\textbf{\Large COUPLED}	& \textbf{\Large DECOUPLED} \\
	\end{minipage}
}
	%\begin{minipage}{0.32\linewidth}
	%	Using Cloudnet liquid and ice water content retrievals (lwc \& iwc, respectively), the total water path for liquid and ice (LWP \& IWP) are calculated by integrating their water content from cloud base to cloud top for individual cloud layers (Fig.~\ref{fig:lwp}). 
	%	\begin{center}             
	%		\includegraphics[width=.91\linewidth]{IWP_LWP_timeseries_20191118}
	%		\captionof{figure}{}
	%		\label{fig:lwp}
	%	\end{center}
	%\end{minipage}
	%&
\end{tabular}
}
%%%%%%%%%%%%%%%%%%%%%%%%%%%%%%%%%%%%%%%%%%%%%%%%%%%%%%%%%%%%%%%%%%%%%%%%%%%%%%
\headerbox{4.- Statistical Results}{name=stats,column=3,span=1,row=0}{
	%%%%%%%%%%%%%%%%%%%%%%%%%%%%%%%%%%%%%%%%%%%%%%%%%%%%%%%%%%%%%%%%%%%%%%%%%%%%%%
Based on the analysis in Box 2 \& 3 and applied to the whole wintertime data from Nov 2019 to April 2020, the following results are found:\\
\colouredcircle Cloud coupling classification: criteria based on the virtual potential temperature $\theta_v$ and location of maximum $\nabla~WVT$ below PBLH. The $\theta_v$ is analyzed to classify cases where the WVT is coupled or decoupled to the cloud mixing layer.\\
\begin{minipage}{0.93\linewidth}
	\centering
	[a]{\includegraphics[width=.3\linewidth]{CBH_coup-decoup}}\hspace{1cm}
	[b]{\includegraphics[width=.3\linewidth]{deltaH_coup-decoup}}
	[c]{\includegraphics[width=.3\linewidth]{CTT_coup-decoup}}\hspace{1cm}
	[d]{\includegraphics[width=.37\linewidth]{occurences_coup-decoup}}
	\captionsetup{width=0.93\linewidth}
	\captionof{figure}{PDF for cloud-base height [a], -layer thickness [b], -top temperature [c], and [d] number of occurrences of coupled (red) and decoupled (blue) observations.}
	\label{fig:stats_codeco}
\end{minipage}
\begin{minipage}{0.9\linewidth}
	\centering
	\includegraphics[width=.9\linewidth]{LWP_IWP_vs_LF_SIC}
	\captionsetup{width=0.93\linewidth}
	\captionof{figure}{Statistics for LWP vs LF (top left) and LWP vs SIC (top right), and IWP vs LF (bottom left) and IWP vs SIC (bottom right)}
	\label{fig:lf_sic_lwp_iwp}
\end{minipage}
}

%%%%%%%%%%%%%%%%%%%%%%%%%%%%%%%%%%%%%%%%%%%%%%%%%%%%%%%%%%%%%%%%%%%%%%%%%%%%%%
\headerbox{5.- Conclusions}{name=conclusions,column=3,span=1,below=stats}{
  %%%%%%%%%%%%%%%%%%%%%%%%%%%%%%%%%%%%%%%%%%%%%%%%%%%%%%%%%%%%%%%%%%%%%%%%%%%%%%
{\normalsize
\begin{itemize}
		\item Relating cloud observations with LF upwind with water vapour transport as conveying mechanism for the coupling as a plausible approach,
		\item When Leads are present, coupled clouds with larger LWP are more frequent,
		\item Increasing of LWP with LF (decreasing of SIC),
		\item Ice water shows no clear relation with sea ice LF or SIC,
		\item Cloud top temperature is warmer and cloud layer thicker for coupled obs.,
		%\item Larger LF are acompanied by warmer cloud top, indicating a higher moisture content for those cases,
		\item Confirmation that coupled clouds are mainly low level clouds (similar for Utiqiaǵvik, Alaska~\refp{bib:garfias2021}),
		%\item SAR based LF at 700~m spatial resolution are crucial, for this case study the same results cannot be reproduce by using passive SIC retrievals at 1~km resolution (see different pattern of SIC and LF in Fig.~\ref{fig:wdirwvt},
\end{itemize}
}
% \sc (this make cool big font)!
}

%%%%%%%%%%%%%%%%%%%%%%%%%%%%%%%%%%%%%%%%%%%%%%%%%%%%%%%%%%%%%%%%%%%%%%%%%%%%%%
\headerbox{References }{name=references,column=1, span=2, above=bottom}{
%%%%%%%%%%%%%%%%%%%%%%%%%%%%%%%%%%%%%%%%%%%%%%%%%%%%%%%%%%%%%%%%%%%%%%%%%%%%%%
%\colouredcircle \hspace{2em}
\hspace{-2.4em}
\begin{minipage}{0.99\linewidth}
{\smaller
	\begin{enumerate}
		\setlength\itemsep{0.08em}
		\item \label{bib:Shupe2022}Shupe, M. et al. "Overview of the \mosaic
		expedition~{Atmosphere}, Elementa: Science of the Anthropocene, doi:10.1525/elementa.2021.00060, (2022).
		\item \label{bib:Ebell2022}Ebell, K. et al. "Temperature and humidity profiles, integrated water vapour and liquid water path derived from the HATPRO microwave radiometer onboard the Polarstern during the MOSAiC expedition", doi:10.1594/PANGAEA.941389, (2022).
		\item \label{bib:Tukiainen2020}Tukiainen, S. et al. "CloudnetPy: A Python package for processing cloud remote sensing data", JOSS, doi:10.21105/joss.02123, (2020).
		\item \label{bib:vonAlbedyll2021}von~Albedyll, L. et al. "Linking sea ice deformation to ice thickness redistribution using high-resolution satellite and airborne observations", The Cryosphere, doi:10.5194/tc-15-2167-2021, (2021).
		\item \label{bib:Ludwig2021}Ludwig, V., et al. "Evaluation of a {New} {Merged} {SIC} {Dataset} at 1~km {Resolution} from MODIS and {AMSR2} {Data} in the {Arctic}", Remote Sensing, doi:10.3390/rs12193183, (2020).
		\item \label{bib:garfias2021}Saavedra~Garfias, P., Kalesse-Los, H. et al. "Climatology of clouds containing supercooled liquid in the Western and Central Arctic". AGU (2021), ESSOAR. DOI10.1002/essoar.10509918.1
		\item \label{bib:jannos2022}Michaelis J, Lüpkes C. "The Impact of Lead Patterns on Mean Profiles of Wind, Temperature, and Turbulent Fluxes in the Atmospheric Boundary Layer over Sea Ice". Atmosphere. (2022) https://doi.org/10.3390/atmos13010148 
	\end{enumerate}
}	
\vspace{-0.6em}
\end{minipage}
	%\end{tabular}
	%\vspace{0.5em}
}
%%%%%%%%%%%%%%%%%%%%%%%%%%%%%%%%%%%%%%%%%%%%%%%%%%%%%%%%%%%%%%%%%%%%%%%%%%%%%%
\headerbox{Acknowledgements}{name=acknown,column=3, span=1, above=bottom}{
	%%%%%%%%%%%%%%%%%%%%%%%%%%%%%%%%%%%%%%%%%%%%%%%%%%%%%%%%%%%%%%%%%%%%%%%%%%%%%%
	{\smaller This work is supported by the DFG funded Transregio-project TR-172 "Arctic Amplification $(AC)^3$". Authors thank to DOE ARM program for providing \mosaic data and the \mosaic community. Cloud classification performed with open-source \emph{Cloudnetpy} by ACTRIS and FMI.
	}\\
{\hspace{3cm} \color{white}\smaller Gedruckt im Universitätsrechenzentrum Leipzig}
}
%%%%%%%%%%%%%%%%%%%%%%%%%%%%%%%%%%%%%%%%%%%%%%%%%%%%%%%%%%%%%%%%%%%%%%%%%%%%%%%%%%
%\headerbox{[*] Contact}{name=contact, column=2, below=results}{
%{\Large \color{white}{Email:	pablo.saavedra@uni-leipzig.de}}\\
%\hspace{3em}\includegraphics[scale=0.04]{github_pablosaa.png}\hspace{3em}
%\includegraphics[scale=0.16]{twitter_DunkleWolke.png}
%}

\end{poster}

\end{document}

%%%%%%%%%%%%%%%%%%%%%%%%%%%%%%%%%%%%%%%%%%%%%%%%%%%%%%%%%%%%%%%%%%%%%%%%%%%%%%%
